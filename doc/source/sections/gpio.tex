\newpage
\subsection{GPIO}\label{sec:per_gpio}
The GPIO peripheral implements a general-purpose input/output interface for the system, allowing input and output control of up to 512 individual pins, as well as individual-pin input change notification interrupts for responding to external events.

Externally, the GPIO peripheral provides a virtual bidirectional bus with separate input, output, and pin drive/direction signals. These are intended to be combined into a true bidirectional bus on the system level.

In the \ttt{edusoc\_basic} configuration, there is no bidirectional GPIO bus, but individual pins are assigned for use with the Arty S7 LEDs, buttons, and switches, with pin directions and change notifications configured accordingly by the provided bootloader. See Appendix \ref{sec:basic} for more information.

Internally, the GPIO pins are divided into 32 pin (32 bit) ports, with a set of registers to control each port. The total number of ports is configurable between 1 port (32 pins total) and 16 ports (512 pins total) (see Section \ref{sec:config}), where the default port count is 1.

In the following register descriptions, each type of register is only described once, even if it has multiple copies for the different ports. $i$ is used as a placeholder for the port index in these cases (where the first port is $i = 0$). In register addresses, $i$ should be replaced with the hexadecimal representation of the desired port index.

This peripheral has the following registers:\\
\begin{table}[H]
    \centering
    \begin{tabular}{|c|c|l|}\hline
        Address & Name & Reset Value \\\hline\hline
        \ttt{0x1B001$i$00} & \ttt{GPIO\_PORT\_$i$} & \ttt{0x????????}\footnotemark[1] \\
        \ttt{0x1B001$i$10} & \ttt{GPIO\_LATCH\_$i$} & \ttt{0x00000000} \\
        \ttt{0x1B001$i$20} & \ttt{GPIO\_DIR\_$i$} & \ttt{0x00000000} \\
        \ttt{0x1B001$i$30} & \ttt{GPIO\_CNR\_$i$} & \ttt{0x00000000} \\
        \ttt{0x1B001$i$40} & \ttt{GPIO\_CNF\_$i$} & \ttt{0x00000000} \\
        \ttt{0x1B001$i$50} & \ttt{GPIO\_CN\_STATE\_$i$} & \ttt{0x00000000} \\
        \ttt{0x1B0010F0} & \ttt{GPIO\_INT\_STATUS} & \ttt{0x00000000} \\\hline
    \end{tabular}
    \caption{GPIO Registers}
    \label{tab:gpio_regs}
\end{table}
\footnotetext[1]{Since the read value of this register depends on external pin states, there is no defined reset value.}

\newpage
\subsubsection{\ttt{GPIO\_PORT\_$i$} Register}
\vspace{-3mm}
Address: \ttt{0x1B001$i$00}\hfill
Reset value: \ttt{0x????????}\footnotemark[1]\\[-1mm]
\begin{table}[H]
    \centering
    \renewcommand{\arraystretch}{1.1}
    \renewcommand{\extrarowheight}{-1.5mm}
    \begin{tabular}{|c||B|B|B|B|B|B|B|>{\arraybackslash}B|}\hline
        \rule{0pt}{12pt}Bit Range & \multicolumn{8}{c|}{Bits (highest to lowest)} \\\hline\hline
        \br{31:24} & \bd{R/W-x} & \bd{R/W-x} & \bd{R/W-x} & \bd{R/W-x} & \bd{R/W-x} & \bd{R/W-x} & \bd{R/W-x} & \bd{R/W-x} \\\cline{2-9}
        \rule{0pt}{12pt} & \multicolumn{8}{c|}{\ttt{PORT\_$i$}[31:24]} \\\hline
        \br{23:16} & \bd{R/W-x} & \bd{R/W-x} & \bd{R/W-x} & \bd{R/W-x} & \bd{R/W-x} & \bd{R/W-x} & \bd{R/W-x} & \bd{R/W-x} \\\cline{2-9}
        \rule{0pt}{12pt} & \multicolumn{8}{c|}{\ttt{PORT\_$i$}[23:16]} \\\hline
        \br{15:8} & \bd{R/W-x} & \bd{R/W-x} & \bd{R/W-x} & \bd{R/W-x} & \bd{R/W-x} & \bd{R/W-x} & \bd{R/W-x} & \bd{R/W-x} \\\cline{2-9}
        \rule{0pt}{12pt} & \multicolumn{8}{c|}{\ttt{PORT\_$i$}[15:8]} \\\hline
        \br{7:0} & \bd{R/W-x} & \bd{R/W-x} & \bd{R/W-x} & \bd{R/W-x} & \bd{R/W-x} & \bd{R/W-x} & \bd{R/W-x} & \bd{R/W-x} \\\cline{2-9}
        \rule{0pt}{12pt} & \multicolumn{8}{c|}{\ttt{PORT\_$i$}[7:0]} \\\hline
        \multicolumn{9}{c}{} \\\hline
        \multicolumn{2}{|l}{\rule{0pt}{10pt}\footnotesize\textbf{Legend:}} & \multicolumn{2}{l}{\footnotesize R = Readable} & \multicolumn{2}{l}{\footnotesize W = Writable} & \multicolumn{3}{l|}{\footnotesize U = Unimplemented, read as 0} \\
        \multicolumn{2}{|l}{\rule{0pt}{10pt}\footnotesize -n = Initial Value} & \multicolumn{2}{l}{\footnotesize 1 = Set} & \multicolumn{2}{l}{\footnotesize 0 = Cleared} & \multicolumn{3}{l|}{\footnotesize x = Unknown} \\\hline
    \end{tabular}
    \renewcommand{\arraystretch}{1}
    \renewcommand{\extrarowheight}{0mm}
\end{table}
\vspace{-2mm}
\begin{itemize}[leftmargin=18mm,labelsep=3mm,parsep=1.5mm]
    \item[\footnotesize Bit 31-0] \ttt{PORT\_$i$}: GPIO port $i$ pin states.\\{\footnotesize
    0 = Corresponding pin is in a low (0) state.\\
    1 = Corresponding pin is in a high (1) state.}
\end{itemize}
If a pin is configured as an output, the corresponding pin in this register represents its currently driven value.\\
Otherwise, the corresponding pin in this register represents its current input signal value.

Writing to this register is equivalent to writing to the \ttt{GPIO\_LATCH\_$i$} register (see below).

\subsubsection{\ttt{GPIO\_LATCH\_$i$} Register}
\vspace{-3mm}
Address: \ttt{0x1B001$i$10}\hfill
Reset value: \ttt{0x00000000}\\[-1mm]
\begin{table}[H]
    \centering
    \renewcommand{\arraystretch}{1.1}
    \renewcommand{\extrarowheight}{-1.5mm}
    \begin{tabular}{|c||B|B|B|B|B|B|B|>{\arraybackslash}B|}\hline
        \rule{0pt}{12pt}Bit Range & \multicolumn{8}{c|}{Bits (highest to lowest)} \\\hline\hline
        \br{31:24} & \bd{R/W-0} & \bd{R/W-0} & \bd{R/W-0} & \bd{R/W-0} & \bd{R/W-0} & \bd{R/W-0} & \bd{R/W-0} & \bd{R/W-0} \\\cline{2-9}
        \rule{0pt}{12pt} & \multicolumn{8}{c|}{\ttt{LATCH\_$i$}[31:24]} \\\hline
        \br{23:16} & \bd{R/W-0} & \bd{R/W-0} & \bd{R/W-0} & \bd{R/W-0} & \bd{R/W-0} & \bd{R/W-0} & \bd{R/W-0} & \bd{R/W-0} \\\cline{2-9}
        \rule{0pt}{12pt} & \multicolumn{8}{c|}{\ttt{LATCH\_$i$}[23:16]} \\\hline
        \br{15:8} & \bd{R/W-0} & \bd{R/W-0} & \bd{R/W-0} & \bd{R/W-0} & \bd{R/W-0} & \bd{R/W-0} & \bd{R/W-0} & \bd{R/W-0} \\\cline{2-9}
        \rule{0pt}{12pt} & \multicolumn{8}{c|}{\ttt{LATCH\_$i$}[15:8]} \\\hline
        \br{7:0} & \bd{R/W-0} & \bd{R/W-0} & \bd{R/W-0} & \bd{R/W-0} & \bd{R/W-0} & \bd{R/W-0} & \bd{R/W-0} & \bd{R/W-0} \\\cline{2-9}
        \rule{0pt}{12pt} & \multicolumn{8}{c|}{\ttt{LATCH\_$i$}[7:0]} \\\hline
        \multicolumn{9}{c}{} \\\hline
        \multicolumn{2}{|l}{\rule{0pt}{10pt}\footnotesize\textbf{Legend:}} & \multicolumn{2}{l}{\footnotesize R = Readable} & \multicolumn{2}{l}{\footnotesize W = Writable} & \multicolumn{3}{l|}{\footnotesize U = Unimplemented, read as 0} \\
        \multicolumn{2}{|l}{\rule{0pt}{10pt}\footnotesize -n = Initial Value} & \multicolumn{2}{l}{\footnotesize 1 = Set} & \multicolumn{2}{l}{\footnotesize 0 = Cleared} & \multicolumn{3}{l|}{\footnotesize x = Unknown} \\\hline
    \end{tabular}
    \renewcommand{\arraystretch}{1}
    \renewcommand{\extrarowheight}{0mm}
\end{table}
\vspace{-2mm}
\begin{itemize}[leftmargin=18mm,labelsep=3mm,parsep=1.5mm]
    \item[\footnotesize Bit 31-0] \ttt{LATCH\_$i$}: GPIO port $i$ output latch states.\\{\footnotesize
    0 = Corresponding pin latch is in a low (0) state.\\
    1 = Corresponding pin latch is in a high (1) state.}
\end{itemize}
\largepage
The latch state of a pin, defined in this register, determines the logic level that will be driven on the pin if it is configured as an output pin. For input-configured pins, the latch value is ignored.

\subsubsection{\ttt{GPIO\_DIR\_$i$} Register}
\vspace{-3mm}
Address: \ttt{0x1B001$i$20}\hfill
Reset value: \ttt{0x00000000}\\[-1mm]
\begin{table}[H]
    \centering
    \renewcommand{\arraystretch}{1.1}
    \renewcommand{\extrarowheight}{-1.5mm}
    \begin{tabular}{|c||B|B|B|B|B|B|B|>{\arraybackslash}B|}\hline
        \rule{0pt}{12pt}Bit Range & \multicolumn{8}{c|}{Bits (highest to lowest)} \\\hline\hline
        \br{31:24} & \bd{R/W-0} & \bd{R/W-0} & \bd{R/W-0} & \bd{R/W-0} & \bd{R/W-0} & \bd{R/W-0} & \bd{R/W-0} & \bd{R/W-0} \\\cline{2-9}
        \rule{0pt}{12pt} & \multicolumn{8}{c|}{\ttt{DIR\_$i$}[31:24]} \\\hline
        \br{23:16} & \bd{R/W-0} & \bd{R/W-0} & \bd{R/W-0} & \bd{R/W-0} & \bd{R/W-0} & \bd{R/W-0} & \bd{R/W-0} & \bd{R/W-0} \\\cline{2-9}
        \rule{0pt}{12pt} & \multicolumn{8}{c|}{\ttt{DIR\_$i$}[23:16]} \\\hline
        \br{15:8} & \bd{R/W-0} & \bd{R/W-0} & \bd{R/W-0} & \bd{R/W-0} & \bd{R/W-0} & \bd{R/W-0} & \bd{R/W-0} & \bd{R/W-0} \\\cline{2-9}
        \rule{0pt}{12pt} & \multicolumn{8}{c|}{\ttt{DIR\_$i$}[15:8]} \\\hline
        \br{7:0} & \bd{R/W-0} & \bd{R/W-0} & \bd{R/W-0} & \bd{R/W-0} & \bd{R/W-0} & \bd{R/W-0} & \bd{R/W-0} & \bd{R/W-0} \\\cline{2-9}
        \rule{0pt}{12pt} & \multicolumn{8}{c|}{\ttt{DIR\_$i$}[7:0]} \\\hline
        \multicolumn{9}{c}{} \\\hline
        \multicolumn{2}{|l}{\rule{0pt}{10pt}\footnotesize\textbf{Legend:}} & \multicolumn{2}{l}{\footnotesize R = Readable} & \multicolumn{2}{l}{\footnotesize W = Writable} & \multicolumn{3}{l|}{\footnotesize U = Unimplemented, read as 0} \\
        \multicolumn{2}{|l}{\rule{0pt}{10pt}\footnotesize -n = Initial Value} & \multicolumn{2}{l}{\footnotesize 1 = Set} & \multicolumn{2}{l}{\footnotesize 0 = Cleared} & \multicolumn{3}{l|}{\footnotesize x = Unknown} \\\hline
    \end{tabular}
    \renewcommand{\arraystretch}{1}
    \renewcommand{\extrarowheight}{0mm}
\end{table}
\vspace{-2mm}
\begin{itemize}[leftmargin=18mm,labelsep=3mm,parsep=1.5mm]
    \item[\footnotesize Bit 31-0] \ttt{DIR\_$i$}: GPIO port $i$ pin directions.\\{\footnotesize
    0 = Corresponding pin is configured as an input pin.\\
    1 = Corresponding pin is configured as an output pin.}
\end{itemize}

\subsubsection{\ttt{GPIO\_CNR\_$i$} Register}
\vspace{-3mm}
Address: \ttt{0x1B001$i$30}\hfill
Reset value: \ttt{0x00000000}\\[-1mm]
\begin{table}[H]
    \centering
    \renewcommand{\arraystretch}{1.1}
    \renewcommand{\extrarowheight}{-1.5mm}
    \begin{tabular}{|c||B|B|B|B|B|B|B|>{\arraybackslash}B|}\hline
        \rule{0pt}{12pt}Bit Range & \multicolumn{8}{c|}{Bits (highest to lowest)} \\\hline\hline
        \br{31:24} & \bd{R/W-0} & \bd{R/W-0} & \bd{R/W-0} & \bd{R/W-0} & \bd{R/W-0} & \bd{R/W-0} & \bd{R/W-0} & \bd{R/W-0} \\\cline{2-9}
        \rule{0pt}{12pt} & \multicolumn{8}{c|}{\ttt{CNR\_$i$}[31:24]} \\\hline
        \br{23:16} & \bd{R/W-0} & \bd{R/W-0} & \bd{R/W-0} & \bd{R/W-0} & \bd{R/W-0} & \bd{R/W-0} & \bd{R/W-0} & \bd{R/W-0} \\\cline{2-9}
        \rule{0pt}{12pt} & \multicolumn{8}{c|}{\ttt{CNR\_$i$}[23:16]} \\\hline
        \br{15:8} & \bd{R/W-0} & \bd{R/W-0} & \bd{R/W-0} & \bd{R/W-0} & \bd{R/W-0} & \bd{R/W-0} & \bd{R/W-0} & \bd{R/W-0} \\\cline{2-9}
        \rule{0pt}{12pt} & \multicolumn{8}{c|}{\ttt{CNR\_$i$}[15:8]} \\\hline
        \br{7:0} & \bd{R/W-0} & \bd{R/W-0} & \bd{R/W-0} & \bd{R/W-0} & \bd{R/W-0} & \bd{R/W-0} & \bd{R/W-0} & \bd{R/W-0} \\\cline{2-9}
        \rule{0pt}{12pt} & \multicolumn{8}{c|}{\ttt{CNR\_$i$}[7:0]} \\\hline
        \multicolumn{9}{c}{} \\\hline
        \multicolumn{2}{|l}{\rule{0pt}{10pt}\footnotesize\textbf{Legend:}} & \multicolumn{2}{l}{\footnotesize R = Readable} & \multicolumn{2}{l}{\footnotesize W = Writable} & \multicolumn{3}{l|}{\footnotesize U = Unimplemented, read as 0} \\
        \multicolumn{2}{|l}{\rule{0pt}{10pt}\footnotesize -n = Initial Value} & \multicolumn{2}{l}{\footnotesize 1 = Set} & \multicolumn{2}{l}{\footnotesize 0 = Cleared} & \multicolumn{3}{l|}{\footnotesize x = Unknown} \\\hline
    \end{tabular}
    \renewcommand{\arraystretch}{1}
    \renewcommand{\extrarowheight}{0mm}
\end{table}
\vspace{-2mm}
\begin{itemize}[leftmargin=18mm,labelsep=3mm,parsep=1.5mm]
    \item[\footnotesize Bit 31-0] \ttt{CNR\_$i$}: GPIO port $i$ rising edge change notification enable.\\{\footnotesize
    0 = Corresponding pin will not generate change notification interrupts for rising edges.\\
    1 = Corresponding pin will generate change notification interrupts for rising edges (0 $\rightarrow$ 1).}
\end{itemize}

\newpage
\subsubsection{\ttt{GPIO\_CNF\_$i$} Register}
\vspace{-3mm}
Address: \ttt{0x1B001$i$40}\hfill
Reset value: \ttt{0x00000000}\\[-1mm]
\begin{table}[H]
    \centering
    \renewcommand{\arraystretch}{1.1}
    \renewcommand{\extrarowheight}{-1.5mm}
    \begin{tabular}{|c||B|B|B|B|B|B|B|>{\arraybackslash}B|}\hline
        \rule{0pt}{12pt}Bit Range & \multicolumn{8}{c|}{Bits (highest to lowest)} \\\hline\hline
        \br{31:24} & \bd{R/W-0} & \bd{R/W-0} & \bd{R/W-0} & \bd{R/W-0} & \bd{R/W-0} & \bd{R/W-0} & \bd{R/W-0} & \bd{R/W-0} \\\cline{2-9}
        \rule{0pt}{12pt} & \multicolumn{8}{c|}{\ttt{CNF\_$i$}[31:24]} \\\hline
        \br{23:16} & \bd{R/W-0} & \bd{R/W-0} & \bd{R/W-0} & \bd{R/W-0} & \bd{R/W-0} & \bd{R/W-0} & \bd{R/W-0} & \bd{R/W-0} \\\cline{2-9}
        \rule{0pt}{12pt} & \multicolumn{8}{c|}{\ttt{CNF\_$i$}[23:16]} \\\hline
        \br{15:8} & \bd{R/W-0} & \bd{R/W-0} & \bd{R/W-0} & \bd{R/W-0} & \bd{R/W-0} & \bd{R/W-0} & \bd{R/W-0} & \bd{R/W-0} \\\cline{2-9}
        \rule{0pt}{12pt} & \multicolumn{8}{c|}{\ttt{CNF\_$i$}[15:8]} \\\hline
        \br{7:0} & \bd{R/W-0} & \bd{R/W-0} & \bd{R/W-0} & \bd{R/W-0} & \bd{R/W-0} & \bd{R/W-0} & \bd{R/W-0} & \bd{R/W-0} \\\cline{2-9}
        \rule{0pt}{12pt} & \multicolumn{8}{c|}{\ttt{CNF\_$i$}[7:0]} \\\hline
        \multicolumn{9}{c}{} \\\hline
        \multicolumn{2}{|l}{\rule{0pt}{10pt}\footnotesize\textbf{Legend:}} & \multicolumn{2}{l}{\footnotesize R = Readable} & \multicolumn{2}{l}{\footnotesize W = Writable} & \multicolumn{3}{l|}{\footnotesize U = Unimplemented, read as 0} \\
        \multicolumn{2}{|l}{\rule{0pt}{10pt}\footnotesize -n = Initial Value} & \multicolumn{2}{l}{\footnotesize 1 = Set} & \multicolumn{2}{l}{\footnotesize 0 = Cleared} & \multicolumn{3}{l|}{\footnotesize x = Unknown} \\\hline
    \end{tabular}
    \renewcommand{\arraystretch}{1}
    \renewcommand{\extrarowheight}{0mm}
\end{table}
\vspace{-2mm}
\begin{itemize}[leftmargin=18mm,labelsep=3mm,parsep=1.5mm]
    \item[\footnotesize Bit 31-0] \ttt{CNF\_$i$}: GPIO port $i$ falling edge change notification enable.\\{\footnotesize
    0 = Corresponding pin will not generate change notification interrupts for falling edges.\\
    1 = Corresponding pin will generate change notification interrupts for falling edges (1 $\rightarrow$ 0).}
\end{itemize}

\subsubsection{\ttt{GPIO\_CN\_STATE\_$i$} Register}
\vspace{-3mm}
Address: \ttt{0x1B001$i$50}\hfill
Reset value: \ttt{0x00000000}\\[-1mm]
\begin{table}[H]
    \centering
    \renewcommand{\arraystretch}{1.1}
    \renewcommand{\extrarowheight}{-1.5mm}
    \begin{tabular}{|c||B|B|B|B|B|B|B|>{\arraybackslash}B|}\hline
        \rule{0pt}{12pt}Bit Range & \multicolumn{8}{c|}{Bits (highest to lowest)} \\\hline\hline
        \br{31:24} & \bd{R/C-0} & \bd{R/C-0} & \bd{R/C-0} & \bd{R/C-0} & \bd{R/C-0} & \bd{R/C-0} & \bd{R/C-0} & \bd{R/C-0} \\\cline{2-9}
        \rule{0pt}{12pt} & \multicolumn{8}{c|}{\ttt{CN\_STATE\_$i$}[31:24]} \\\hline
        \br{23:16} & \bd{R/C-0} & \bd{R/C-0} & \bd{R/C-0} & \bd{R/C-0} & \bd{R/C-0} & \bd{R/C-0} & \bd{R/C-0} & \bd{R/C-0} \\\cline{2-9}
        \rule{0pt}{12pt} & \multicolumn{8}{c|}{\ttt{CN\_STATE\_$i$}[23:16]} \\\hline
        \br{15:8} & \bd{R/C-0} & \bd{R/C-0} & \bd{R/C-0} & \bd{R/C-0} & \bd{R/C-0} & \bd{R/C-0} & \bd{R/C-0} & \bd{R/C-0} \\\cline{2-9}
        \rule{0pt}{12pt} & \multicolumn{8}{c|}{\ttt{CN\_STATE\_$i$}[15:8]} \\\hline
        \br{7:0} & \bd{R/C-0} & \bd{R/C-0} & \bd{R/C-0} & \bd{R/C-0} & \bd{R/C-0} & \bd{R/C-0} & \bd{R/C-0} & \bd{R/C-0} \\\cline{2-9}
        \rule{0pt}{12pt} & \multicolumn{8}{c|}{\ttt{CN\_STATE\_$i$}[7:0]} \\\hline
        \multicolumn{9}{c}{} \\\hline
        \multicolumn{2}{|l}{\rule{0pt}{10pt}\footnotesize\textbf{Legend:}} & \multicolumn{2}{l}{\footnotesize R = Readable} & \multicolumn{2}{l}{\footnotesize W = Writable} & \multicolumn{3}{l|}{\footnotesize U = Unimplemented, read as 0} \\
        \multicolumn{2}{|l}{\footnotesize C = Clearable} & \multicolumn{2}{l}{\rule{0pt}{10pt}\footnotesize -n = Initial Value} & \multicolumn{1}{l}{\footnotesize 1 = Set} & \multicolumn{2}{l}{\footnotesize 0 = Cleared} & \multicolumn{2}{l|}{\footnotesize x = Unknown} \\\hline
    \end{tabular}
    \renewcommand{\arraystretch}{1}
    \renewcommand{\extrarowheight}{0mm}
\end{table}
\vspace{-2mm}
\begin{itemize}[leftmargin=18mm,labelsep=3mm,parsep=1.5mm]
    \item[\footnotesize Bit 31-0] \ttt{CN\_STATE\_$i$}: GPIO port $i$ edge change notification state.\\{\footnotesize
    0 = Corresponding pin has not generated a change notification interrupt.\\
    1 = Corresponding pin has generated a change notification interrupt (i.\,e. it has experienced a signal edge with enabled change notification).}
\end{itemize}
Bits in this register may be cleared (set to 0) using memory writes. Attempting to set bits to 1 using memory writes will have no effect.

\subsubsection{\ttt{GPIO\_INT\_STATUS} Register}
\vspace{-3mm}
Address: \ttt{0x1B0010F0}\hfill
Reset value: \ttt{0x00000000}\\[-1mm]
\begin{table}[H]
    \centering
    \renewcommand{\arraystretch}{1.1}
    \renewcommand{\extrarowheight}{-1.5mm}
    \begin{tabular}{|c||B|B|B|B|B|B|B|>{\arraybackslash}B|}\hline
        \rule{0pt}{12pt}Bit Range & \multicolumn{8}{c|}{Bits (highest to lowest)} \\\hline\hline
        \br{31:24} & \bd{U-0} & \bd{U-0} & \bd{U-0} & \bd{U-0} & \bd{U-0} & \bd{U-0} & \bd{U-0} & \bd{U-0} \\\cline{2-9}
        \rule{0pt}{12pt} & \ub & \ub & \ub & \ub & \ub & \ub & \ub & \ub \\\hline
        \br{23:16} & \bd{U-0} & \bd{U-0} & \bd{U-0} & \bd{U-0} & \bd{U-0} & \bd{U-0} & \bd{U-0} & \bd{U-0} \\\cline{2-9}
        \rule{0pt}{12pt} & \ub & \ub & \ub & \ub & \ub & \ub & \ub & \ub \\\hline
        \br{15:8} & \bd{R/C-0} & \bd{R/C-0} & \bd{R/C-0} & \bd{R/C-0} & \bd{R/C-0} & \bd{R/C-0} & \bd{R/C-0} & \bd{R/C-0} \\\cline{2-9}
        \rule{0pt}{12pt} & \multicolumn{8}{c|}{\ttt{INT\_STATUS}[15:8]} \\\hline
        \br{7:0} & \bd{R/C-0} & \bd{R/C-0} & \bd{R/C-0} & \bd{R/C-0} & \bd{R/C-0} & \bd{R/C-0} & \bd{R/C-0} & \bd{R/C-0} \\\cline{2-9}
        \rule{0pt}{12pt} & \multicolumn{8}{c|}{\ttt{INT\_STATUS}[7:0]} \\\hline
        \multicolumn{9}{c}{} \\\hline
        \multicolumn{2}{|l}{\rule{0pt}{10pt}\footnotesize\textbf{Legend:}} & \multicolumn{2}{l}{\footnotesize R = Readable} & \multicolumn{2}{l}{\footnotesize W = Writable} & \multicolumn{3}{l|}{\footnotesize U = Unimplemented, read as 0} \\
        \multicolumn{2}{|l}{\footnotesize C = Clearable} & \multicolumn{2}{l}{\rule{0pt}{10pt}\footnotesize -n = Initial Value} & \multicolumn{1}{l}{\footnotesize 1 = Set} & \multicolumn{2}{l}{\footnotesize 0 = Cleared} & \multicolumn{2}{l|}{\footnotesize x = Unknown} \\\hline
    \end{tabular}
    \renewcommand{\arraystretch}{1}
    \renewcommand{\extrarowheight}{0mm}
\end{table}
\vspace{-2mm}
\begin{itemize}[leftmargin=18mm,labelsep=3mm,parsep=1.5mm]
    \item[\footnotesize Bit 31-16] Unimplemented: Read as 0.
    \item[\footnotesize Bit 15-0] \ttt{INT\_STATUS}: GPIO ports interrupt status.\\{\footnotesize
    0 = Port with corresponding index has not generated a change notification interrupt.\\
    1 = Port with corresponding index has generated a change notification interrupt (i.\,e. one or more of its pins have experienced a signal edge with enabled change notification).}
\end{itemize}
The purpose of this register is to track which ports have generated change notification interrupts, to allow the software to efficiently find which pin is the latest interrupt source.

Bits in this register may be cleared (set to 0) using memory writes. Attempting to set bits to 1 using memory writes will have no effect.
