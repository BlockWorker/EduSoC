% Table helpers for register bit tables
\newcolumntype{B}{>{\centering}m{14mm}} % "Bit" column type, fixed width
\newcommand{\ub}{\cellcolor{gray!25}\textemdash} % "Unused bit" cell
\newcommand{\bd}[1]{\scriptsize#1\vspace{-2.3mm}} % Bit details (read/write ability, reset value, ...)
\newcommand{\br}[1]{\multirow{2}{*}{#1\vspace{-2.3mm}}} % Bit range cell (first column)

\section{Peripherals}\label{sec:peripherals}
EduSoC contains multiple memory-mapped peripheral modules. They are accessible using the main memory bus and have registers to control their function, which are described in the following sections.

All registers in the following peripherals have four separate addresses each (differentiated by the last hexadecimal digit of the address), implementing functions for easy bit manipulations:
\begin{itemize}
    \item \ttt{0x???????0}: Main register. Allows reading the register value and writing a value directly.
    \item \ttt{0x???????4}: SET register. Reads as 0. When written: For every high (1) bit in the written value, the corresponding register bit is set high (1). Low (0) bits in the written value leave the corresponding register bits unchanged.
    \item \ttt{0x???????8}: CLEAR register. Reads as 0. When written: For every high (1) bit in the written value, the corresponding register bit is cleared (set to 0). Low (0) bits in the written value leave the corresponding register bits unchanged.
    \item \ttt{0x???????C}: INVERT register. Reads as 0. When written: For every high (1) bit in the written value, the corresponding register bit is inverted (0 $\leftrightarrow$ 1). Low (0) bits in the written value leave the corresponding register bits unchanged.
\end{itemize}
In the following register descriptions, only the main register address is given, but the other addresses still function as described above, by using the corresponding last address digit.
The only exceptions are register bits which are read-only or may only be written in certain ways - the SET, CLEAR, and INVERT variants of those registers respect these restrictions, and will not perform any unsupported operations.

The register addresses given in the following register descriptions assume the default memory map as described in Section \ref{sec:memorymap}.

\subsection{SoC Control}\label{sec:per_control}
The SoC control peripheral implements functions for controlling the SoC and system as a whole, as well as controlling system interrupts.

It allows software- or UART-initiated resets of the core and SoC, allows the core to be halted, and allows detailed control over interrupts (enabling/disabling interrupts globally, enabling/disabling individual interrupts, reading interrupt status, clearing individual interrupts).

Additionally, a 16 bit core control flag signal is provided (see \ttt{control\_flags} in Section \ref{sec:socinterface}). These flags may have a user-defined meaning and can also be read or written using the memory bus. Unlike most registers, these flags are \textit{not} affected by system resets, retaining their state.\\
For the bootloader code and UART programming utilities provided with EduSoC, only one of these flags (flag 0) has a meaning by default - it serves as an indicator that a program has been loaded into RAM.

This peripheral has the following registers:\\
\begin{table}[H]
    \centering
    \begin{tabular}{|c|c|l|}\hline
        Address & Name & Reset Value \\\hline\hline
        \ttt{0x1B000000} & \ttt{SOCCON\_CONTROL} & \ttt{0x00000008}\footnotemark[1] \\
        \ttt{0x1B000010} & \ttt{SOCCON\_INT\_EN} & \ttt{0x00000000} \\
        \ttt{0x1B000020} & \ttt{SOCCON\_INT\_FLAGS} & \ttt{0x00000000} \\\hline
    \end{tabular}
    \caption{SoC Control Registers}
    \label{tab:soccon_regs}
\end{table}
\footnotetext[1]{Upper 16 bits are initially 0x0000, but retain their values during further resets.}

\subsubsection{\ttt{SOCCON\_CONTROL} Register}
Address: \ttt{0x1B000000}\\
Reset value: \ttt{0x00000008}\footnotemark[1]\\
\begin{table}[H]
    \centering
    \renewcommand{\arraystretch}{1.1}
    \renewcommand{\extrarowheight}{-1.5mm}
    \begin{tabular}{|c||B|B|B|B|B|B|B|>{\arraybackslash}B|}\hline
        \rule{0pt}{12pt}Bit Range & \multicolumn{8}{c|}{Bits (highest to lowest)} \\\hline\hline
        \br{31:24} & \bd{R/W-0} & \bd{R/W-0} & \bd{R/W-0} & \bd{R/W-0} & \bd{R/W-0} & \bd{R/W-0} & \bd{R/W-0} & \bd{R/W-0} \\\cline{2-9}
        \rule{0pt}{12pt} & \multicolumn{8}{c|}{\ttt{control\_flags}[15:8]} \\\hline
        \br{23:16} & \bd{R/W-0} & \bd{R/W-0} & \bd{R/W-0} & \bd{R/W-0} & \bd{R/W-0} & \bd{R/W-0} & \bd{R/W-0} & \bd{R/W-0} \\\cline{2-9}
        \rule{0pt}{12pt} & \multicolumn{8}{c|}{\ttt{control\_flags}[7:0]} \\\hline
        \br{15:8} & \bd{U-0} & \bd{U-0} & \bd{U-0} & \bd{U-0} & \bd{U-0} & \bd{U-0} & \bd{U-0} & \bd{U-0} \\\cline{2-9}
        \rule{0pt}{12pt} & \ub & \ub & \ub & \ub & \ub & \ub & \ub & \ub \\\hline
        \br{7:0} & \bd{U-0} & \bd{U-0} & \bd{U-0} & \bd{U-0} & \bd{R/W-1} & \bd{R/W-0} & \bd{R/W-0} & \bd{R/W-0} \\\cline{2-9}
        \rule{0pt}{12pt} & \ub & \ub & \ub & \ub & \ttt{INTGEN} & \ttt{SOCRES} & \ttt{CORERES} & \ttt{COREHLT} \\\hline
        \multicolumn{9}{c}{} \\\hline
        \multicolumn{2}{|l}{\rule{0pt}{10pt}\footnotesize\textbf{Legend:}} & \multicolumn{2}{l}{\footnotesize R = Readable} & \multicolumn{2}{l}{\footnotesize W = Writable} & \multicolumn{3}{l|}{\footnotesize U = Unimplemented, read as 0} \\
        \multicolumn{2}{|l}{\rule{0pt}{10pt}\footnotesize -n = Initial Value} & \multicolumn{2}{l}{\footnotesize 1 = Set} & \multicolumn{2}{l}{\footnotesize 0 = Cleared} & \multicolumn{3}{l|}{\footnotesize x = Unknown} \\\hline
    \end{tabular}
    \renewcommand{\arraystretch}{1}
    \renewcommand{\extrarowheight}{0mm}
\end{table}
\begin{itemize}[leftmargin=18mm,labelsep=3mm,parsep=1.5mm]
    \item[\footnotesize Bit 31-16] \ttt{control\_flags}: Core control flags, see description above. Reset retains values.
    \item[\footnotesize Bit 15-4] Unimplemented: Read as 0.
    \item[\footnotesize Bit 3] \ttt{INTGEN}: Interrupt global enable.\\{\footnotesize
    0 = No interrupts will be asserted to the core, irrespective of individual interrupt enable settings.\\
    1 = Individually enabled interrupts will be asserted to the core.}
    \item[\footnotesize Bit 2] \ttt{SOCRES}: SoC software reset.\\{\footnotesize
    0 = No effect.\\
    1 = SoC will be reset on the next clock cycle (including this register/bit).}
    \item[\footnotesize Bit 1] \ttt{CORERES}: CPU core reset.\\{\footnotesize
    0 = No effect.\\
    1 = CPU core reset line is asserted high (1), leading to a reset on the next CPU core clock cycle.}
    \item[\footnotesize Bit 0] \ttt{COREHLT}: CPU core halt.\\{\footnotesize
    0 = CPU core clock is running.\\
    1 = CPU core clock is halted. Note that this also prevents synchronous core resets from occurring.}
\end{itemize}

\subsubsection{\ttt{SOCCON\_INT\_EN} Register}
Address: \ttt{0x1B000010}\\
Reset value: \ttt{0x00000000}\\
\begin{table}[H]
    \centering
    \renewcommand{\arraystretch}{1.1}
    \renewcommand{\extrarowheight}{-1.5mm}
    \begin{tabular}{|c||B|B|B|B|B|B|B|>{\arraybackslash}B|}\hline
        \rule{0pt}{12pt}Bit Range & \multicolumn{8}{c|}{Bits (highest to lowest)} \\\hline\hline
        \br{31:24} & \bd{R/W-0} & \bd{R/W-0} & \bd{R/W-0} & \bd{R/W-0} & \bd{R/W-0} & \bd{R/W-0} & \bd{R/W-0} & \bd{R/W-0} \\\cline{2-9}
        \rule{0pt}{12pt} & \multicolumn{8}{c|}{\ttt{INT\_ENABLE}[31:24]} \\\hline
        \br{23:16} & \bd{R/W-0} & \bd{R/W-0} & \bd{R/W-0} & \bd{R/W-0} & \bd{R/W-0} & \bd{R/W-0} & \bd{R/W-0} & \bd{R/W-0} \\\cline{2-9}
        \rule{0pt}{12pt} & \multicolumn{8}{c|}{\ttt{INT\_ENABLE}[23:16]} \\\hline
        \br{15:8} & \bd{R/W-0} & \bd{R/W-0} & \bd{R/W-0} & \bd{R/W-0} & \bd{R/W-0} & \bd{R/W-0} & \bd{R/W-0} & \bd{R/W-0} \\\cline{2-9}
        \rule{0pt}{12pt} & \multicolumn{8}{c|}{\ttt{INT\_ENABLE}[15:8]} \\\hline
        \br{7:0} & \bd{R/W-0} & \bd{R/W-0} & \bd{R/W-0} & \bd{R/W-0} & \bd{R/W-0} & \bd{R/W-0} & \bd{R/W-0} & \bd{R/W-0} \\\cline{2-9}
        \rule{0pt}{12pt} & \multicolumn{8}{c|}{\ttt{INT\_ENABLE}[7:0]} \\\hline
        \multicolumn{9}{c}{} \\\hline
        \multicolumn{2}{|l}{\rule{0pt}{10pt}\footnotesize\textbf{Legend:}} & \multicolumn{2}{l}{\footnotesize R = Readable} & \multicolumn{2}{l}{\footnotesize W = Writable} & \multicolumn{3}{l|}{\footnotesize U = Unimplemented, read as 0} \\
        \multicolumn{2}{|l}{\rule{0pt}{10pt}\footnotesize -n = Initial Value} & \multicolumn{2}{l}{\footnotesize 1 = Set} & \multicolumn{2}{l}{\footnotesize 0 = Cleared} & \multicolumn{3}{l|}{\footnotesize x = Unknown} \\\hline
    \end{tabular}
    \renewcommand{\arraystretch}{1}
    \renewcommand{\extrarowheight}{0mm}
\end{table}
\begin{itemize}[leftmargin=18mm,labelsep=3mm,parsep=1.5mm]
    \item[\footnotesize Bit 31-0] \ttt{INT\_ENABLE}: Individual interrupt enable.\\{\footnotesize
    0 = Interrupt with the corresponding index will not be asserted to the core.\\
    1 = Interrupt with the corresponding index will be asserted to the core whenever it occurs.}
\end{itemize}
Each bit corresponds to one interrupt ID, e.\,g. bit 0 corresponds to interrupt ID 0.

When enabling an interrupt, if it has occurred in the past (i.\,e. its flag is 1, see \ttt{SOCCON\_INT\_FLAGS}), it will immediately be asserted to the core.

\newpage
\subsubsection{\ttt{SOCCON\_INT\_FLAGS} Register}
Address: \ttt{0x1B000020}\\
Reset value: \ttt{0x00000000}\\
\begin{table}[H]
    \centering
    \renewcommand{\arraystretch}{1.1}
    \renewcommand{\extrarowheight}{-1.5mm}
    \begin{tabular}{|c||B|B|B|B|B|B|B|>{\arraybackslash}B|}\hline
        \rule{0pt}{12pt}Bit Range & \multicolumn{8}{c|}{Bits (highest to lowest)} \\\hline\hline
        \br{31:24} & \bd{R/C-0} & \bd{R/C-0} & \bd{R/C-0} & \bd{R/C-0} & \bd{R/C-0} & \bd{R/C-0} & \bd{R/C-0} & \bd{R/C-0} \\\cline{2-9}
        \rule{0pt}{12pt} & \multicolumn{8}{c|}{\ttt{INT\_FLAGS}[31:24]} \\\hline
        \br{23:16} & \bd{R/C-0} & \bd{R/C-0} & \bd{R/C-0} & \bd{R/C-0} & \bd{R/C-0} & \bd{R/C-0} & \bd{R/C-0} & \bd{R/C-0} \\\cline{2-9}
        \rule{0pt}{12pt} & \multicolumn{8}{c|}{\ttt{INT\_FLAGS}[23:16]} \\\hline
        \br{15:8} & \bd{R/C-0} & \bd{R/C-0} & \bd{R/C-0} & \bd{R/C-0} & \bd{R/C-0} & \bd{R/C-0} & \bd{R/C-0} & \bd{R/C-0} \\\cline{2-9}
        \rule{0pt}{12pt} & \multicolumn{8}{c|}{\ttt{INT\_FLAGS}[15:8]} \\\hline
        \br{7:0} & \bd{R/C-0} & \bd{R/C-0} & \bd{R/C-0} & \bd{R/C-0} & \bd{R/C-0} & \bd{R/C-0} & \bd{R/C-0} & \bd{R/C-0} \\\cline{2-9}
        \rule{0pt}{12pt} & \multicolumn{8}{c|}{\ttt{INT\_FLAGS}[7:0]} \\\hline
        \multicolumn{9}{c}{} \\\hline
        \multicolumn{2}{|l}{\rule{0pt}{10pt}\footnotesize\textbf{Legend:}} & \multicolumn{2}{l}{\footnotesize R = Readable} & \multicolumn{2}{l}{\footnotesize W = Writable} & \multicolumn{3}{l|}{\footnotesize U = Unimplemented, read as 0} \\
        \multicolumn{2}{|l}{\footnotesize C = Clearable} & \multicolumn{2}{l}{\rule{0pt}{10pt}\footnotesize -n = Initial Value} & \multicolumn{1}{l}{\footnotesize 1 = Set} & \multicolumn{2}{l}{\footnotesize 0 = Cleared} & \multicolumn{2}{l|}{\footnotesize x = Unknown} \\\hline
    \end{tabular}
    \renewcommand{\arraystretch}{1}
    \renewcommand{\extrarowheight}{0mm}
\end{table}
\begin{itemize}[leftmargin=18mm,labelsep=3mm,parsep=1.5mm]
    \item[\footnotesize Bit 31-0] \ttt{INT\_FLAGS}: Interrupt occurred flags.\\{\footnotesize
    0 = Interrupt with the corresponding index has not occurred.\\
    1 = Interrupt with the corresponding index has occurred and will be asserted to the core if it is enabled.}
\end{itemize}
Each bit corresponds to one interrupt ID, e.\,g. bit 0 corresponds to interrupt ID 0.

Flags/bits in this register may be cleared (set to 0) using memory writes, which will clear the corresponding interrupts and stop them from being asserted until they reoccur. Attempting to set bits to 1 using memory writes will have no effect.

\subsection{GPIO}\label{sec:per_gpio}

\subsection{Timer}\label{sec:per_timer}

\newpage
\subsection{PWM}\label{sec:per_pwm}
The PWM peripheral implements configurable pulse width modulated signal generation. It consists of up to 16 independent PWM modules (configurable, see Section \ref{sec:config}), where the default PWM module count is 6.

In the \ttt{edusoc\_basic} configuration, the 6 PWM modules are connected to the two RGB LEDs on the Arty S7 board, as a way to generate any desired LED colors. However, the corresponding timers and PWM modules are not automatically configured by the provided bootloader. See Appendix \ref{sec:basic} for more information.

Each PWM module depends on one timer from the timer peripheral (see Section \ref{sec:per_timer}). The PWM signal period is equal to the timer's period, and the timer must be enabled and running continuously for correct PWM signal generation. Multiple PWM modules can use the same timer.

The modulation value (pulse width) of a PWM module is controlled by its value register, which determines the absolute width of each pulse (in clock cycles). For a value of 0, the signal is constantly low (0), for a value greater than or equal to the corresponding timer period, the signal is constantly high (1).

The value register cannot be modified directly, instead, each module has a ``next value'' register which can be written to, and will be copied to the value register at the start of the following PWM/timer period.

In the following register descriptions, each type of register is only described once, even if it has multiple copies for the different PWM modules. $i$ is used as a placeholder for the module index in these cases (where the first module is $i = 0$). In register addresses, $i$ should be replaced with the hexadecimal representation of the desired module index.

This peripheral has the following registers:\\
\begin{table}[H]
    \centering
    \begin{tabular}{|c|c|l|}\hline
        Address & Name & Reset Value \\\hline\hline
        \ttt{0x1B003$i$00} & \ttt{PWM\_CONTROL\_$i$} & \ttt{0x00000000} \\
        \ttt{0x1B003$i$10} & \ttt{PWM\_VALUE\_$i$} & \ttt{0x00000000} \\
        \ttt{0x1B003$i$20} & \ttt{PWM\_NEXT\_VALUE\_$i$} & \ttt{0x00000000} \\\hline
    \end{tabular}
    \caption{PWM Registers}
    \label{tab:pwm_regs}
\end{table}

\newpage
\subsubsection{\ttt{PWM\_CONTROL\_$i$} Register}
\vspace{-3mm}
Address: \ttt{0x1B003$i$00}\hfill
Reset value: \ttt{0x00000000}\\[-1mm]
\begin{table}[H]
    \centering
    \renewcommand{\arraystretch}{1.1}
    \renewcommand{\extrarowheight}{-1.5mm}
    \begin{tabular}{|c||B|B|B|B|B|B|B|>{\arraybackslash}B|}\hline
        \rule{0pt}{12pt}Bit Range & \multicolumn{8}{c|}{Bits (highest to lowest)} \\\hline\hline
        \br{31:24} & \bd{U-0} & \bd{U-0} & \bd{U-0} & \bd{U-0} & \bd{U-0} & \bd{U-0} & \bd{U-0} & \bd{U-0} \\\cline{2-9}
        \rule{0pt}{12pt} & \ub & \ub & \ub & \ub & \ub & \ub & \ub & \ub \\\hline
        \br{23:16} & \bd{U-0} & \bd{U-0} & \bd{U-0} & \bd{U-0} & \bd{U-0} & \bd{U-0} & \bd{U-0} & \bd{U-0} \\\cline{2-9}
        \rule{0pt}{12pt} & \ub & \ub & \ub & \ub & \ub & \ub & \ub & \ub \\\hline
        \br{15:8} & \bd{U-0} & \bd{U-0} & \bd{U-0} & \bd{U-0} & \bd{U-0} & \bd{U-0} & \bd{U-0} & \bd{R/W-0} \\\cline{2-9}
        \rule{0pt}{12pt} & \ub & \ub & \ub & \ub & \ub & \ub & \ub & \ttt{ENABLE} \\\hline
        \br{7:0} & \bd{U-0} & \bd{U-0} & \bd{U-0} & \bd{U-0} & \bd{R/W-0} & \bd{R/W-0} & \bd{R/W-0} & \bd{R/W-0} \\\cline{2-9}
        \rule{0pt}{12pt} & \ub & \ub & \ub & \ub & \multicolumn{4}{c|}{\ttt{TIMER\_ID}} \\\hline
        \multicolumn{9}{c}{} \\\hline
        \multicolumn{2}{|l}{\rule{0pt}{10pt}\footnotesize\textbf{Legend:}} & \multicolumn{2}{l}{\footnotesize R = Readable} & \multicolumn{2}{l}{\footnotesize W = Writable} & \multicolumn{3}{l|}{\footnotesize U = Unimplemented, read as 0} \\
        \multicolumn{2}{|l}{\rule{0pt}{10pt}\footnotesize -n = Initial Value} & \multicolumn{2}{l}{\footnotesize 1 = Set} & \multicolumn{2}{l}{\footnotesize 0 = Cleared} & \multicolumn{3}{l|}{\footnotesize x = Unknown} \\\hline
    \end{tabular}
    \renewcommand{\arraystretch}{1}
    \renewcommand{\extrarowheight}{0mm}
\end{table}
\vspace{-2mm}
\begin{itemize}[leftmargin=18mm,labelsep=3mm,parsep=1.5mm]
    \item[\footnotesize Bit 31-9] Unimplemented: Read as 0.
    \item[\footnotesize Bit 8] \ttt{ENABLE}: PWM module enable.\\{\footnotesize
    0 = PWM module $i$ is disabled, its output is constantly low (0).\\
    1 = PWM module $i$ is enabled and outputting a PWM signal.}
    \item[\footnotesize Bit 7-4] Unimplemented: Read as 0.
    \item[\footnotesize Bit 3-0] \ttt{TIMER\_ID}: Index of the timer module used for PWM module $i$. See Section \ref{sec:per_timer}.
\end{itemize}

\subsubsection{\ttt{PWM\_VALUE\_$i$} Register}
\vspace{-3mm}
Address: \ttt{0x1B003$i$10}\hfill
Reset value: \ttt{0x00000000}\\[-1mm]
\begin{table}[H]
    \centering
    \renewcommand{\arraystretch}{1.1}
    \renewcommand{\extrarowheight}{-1.5mm}
    \begin{tabular}{|c||B|B|B|B|B|B|B|>{\arraybackslash}B|}\hline
        \rule{0pt}{12pt}Bit Range & \multicolumn{8}{c|}{Bits (highest to lowest)} \\\hline\hline
        \br{31:24} & \bd{R-0} & \bd{R-0} & \bd{R-0} & \bd{R-0} & \bd{R-0} & \bd{R-0} & \bd{R-0} & \bd{R-0} \\\cline{2-9}
        \rule{0pt}{12pt} & \multicolumn{8}{c|}{\ttt{VALUE\_$i$}[31:24]} \\\hline
        \br{23:16} & \bd{R-0} & \bd{R-0} & \bd{R-0} & \bd{R-0} & \bd{R-0} & \bd{R-0} & \bd{R-0} & \bd{R-0} \\\cline{2-9}
        \rule{0pt}{12pt} & \multicolumn{8}{c|}{\ttt{VALUE\_$i$}[23:16]} \\\hline
        \br{15:8} & \bd{R-0} & \bd{R-0} & \bd{R-0} & \bd{R-0} & \bd{R-0} & \bd{R-0} & \bd{R-0} & \bd{R-0} \\\cline{2-9}
        \rule{0pt}{12pt} & \multicolumn{8}{c|}{\ttt{VALUE\_$i$}[15:8]} \\\hline
        \br{7:0} & \bd{R-0} & \bd{R-0} & \bd{R-0} & \bd{R-0} & \bd{R-0} & \bd{R-0} & \bd{R-0} & \bd{R-0} \\\cline{2-9}
        \rule{0pt}{12pt} & \multicolumn{8}{c|}{\ttt{VALUE\_$i$}[7:0]} \\\hline
        \multicolumn{9}{c}{} \\\hline
        \multicolumn{2}{|l}{\rule{0pt}{10pt}\footnotesize\textbf{Legend:}} & \multicolumn{2}{l}{\footnotesize R = Readable} & \multicolumn{2}{l}{\footnotesize W = Writable} & \multicolumn{3}{l|}{\footnotesize U = Unimplemented, read as 0} \\
        \multicolumn{2}{|l}{\rule{0pt}{10pt}\footnotesize -n = Initial Value} & \multicolumn{2}{l}{\footnotesize 1 = Set} & \multicolumn{2}{l}{\footnotesize 0 = Cleared} & \multicolumn{3}{l|}{\footnotesize x = Unknown} \\\hline
    \end{tabular}
    \renewcommand{\arraystretch}{1}
    \renewcommand{\extrarowheight}{0mm}
\end{table}
\vspace{-2mm}
\begin{itemize}[leftmargin=18mm,labelsep=3mm,parsep=1.5mm]
    \item[\footnotesize Bit 31-0] \ttt{VALUE\_$i$}: PWM module $i$ pulse width in clock cycles.
\end{itemize}
For a value of 0, the PWM signal is constantly low (0), for a value greater than or equal to the corresponding timer period, the PWM signal is constantly high (1).

This register is read-only, attempting to write to it will have no effect. For setting the value, the \ttt{PWM\_NEXT\_VALUE\_$i$} register should be used instead (see below).

\newpage
\subsubsection{\ttt{PWM\_NEXT\_VALUE\_$i$} Register}
\vspace{-3mm}
Address: \ttt{0x1B003$i$20}\hfill
Reset value: \ttt{0x00000000}\\[-1mm]
\begin{table}[H]
    \centering
    \renewcommand{\arraystretch}{1.1}
    \renewcommand{\extrarowheight}{-1.5mm}
    \begin{tabular}{|c||B|B|B|B|B|B|B|>{\arraybackslash}B|}\hline
        \rule{0pt}{12pt}Bit Range & \multicolumn{8}{c|}{Bits (highest to lowest)} \\\hline\hline
        \br{31:24} & \bd{R/W-0} & \bd{R/W-0} & \bd{R/W-0} & \bd{R/W-0} & \bd{R/W-0} & \bd{R/W-0} & \bd{R/W-0} & \bd{R/W-0} \\\cline{2-9}
        \rule{0pt}{12pt} & \multicolumn{8}{c|}{\ttt{NEXT\_VALUE\_$i$}[31:24]} \\\hline
        \br{23:16} & \bd{R/W-0} & \bd{R/W-0} & \bd{R/W-0} & \bd{R/W-0} & \bd{R/W-0} & \bd{R/W-0} & \bd{R/W-0} & \bd{R/W-0} \\\cline{2-9}
        \rule{0pt}{12pt} & \multicolumn{8}{c|}{\ttt{NEXT\_VALUE\_$i$}[23:16]} \\\hline
        \br{15:8} & \bd{R/W-0} & \bd{R/W-0} & \bd{R/W-0} & \bd{R/W-0} & \bd{R/W-0} & \bd{R/W-0} & \bd{R/W-0} & \bd{R/W-0} \\\cline{2-9}
        \rule{0pt}{12pt} & \multicolumn{8}{c|}{\ttt{NEXT\_VALUE\_$i$}[15:8]} \\\hline
        \br{7:0} & \bd{R/W-0} & \bd{R/W-0} & \bd{R/W-0} & \bd{R/W-0} & \bd{R/W-0} & \bd{R/W-0} & \bd{R/W-0} & \bd{R/W-0} \\\cline{2-9}
        \rule{0pt}{12pt} & \multicolumn{8}{c|}{\ttt{NEXT\_VALUE\_$i$}[7:0]} \\\hline
        \multicolumn{9}{c}{} \\\hline
        \multicolumn{2}{|l}{\rule{0pt}{10pt}\footnotesize\textbf{Legend:}} & \multicolumn{2}{l}{\footnotesize R = Readable} & \multicolumn{2}{l}{\footnotesize W = Writable} & \multicolumn{3}{l|}{\footnotesize U = Unimplemented, read as 0} \\
        \multicolumn{2}{|l}{\rule{0pt}{10pt}\footnotesize -n = Initial Value} & \multicolumn{2}{l}{\footnotesize 1 = Set} & \multicolumn{2}{l}{\footnotesize 0 = Cleared} & \multicolumn{3}{l|}{\footnotesize x = Unknown} \\\hline
    \end{tabular}
    \renewcommand{\arraystretch}{1}
    \renewcommand{\extrarowheight}{0mm}
\end{table}
\vspace{-2mm}
\begin{itemize}[leftmargin=18mm,labelsep=3mm,parsep=1.5mm]
    \item[\footnotesize Bit 31-0] \ttt{NEXT\_VALUE\_$i$}: PWM module $i$ next pulse width in clock cycles.
\end{itemize}
The value in this register will be copied to \ttt{PWM\_VALUE\_$i$} at the start of each PWM/timer period.
