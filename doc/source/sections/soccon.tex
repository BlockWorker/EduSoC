\newpage
\subsection{SoC Control}\label{sec:per_control}
The SoC control peripheral implements functions for controlling the SoC and system as a whole, as well as controlling system interrupts.

It allows software- or UART-initiated resets of the core and SoC, allows the core to be halted, and allows detailed control over interrupts (enabling/disabling interrupts globally, enabling/disabling individual interrupts, reading interrupt status, clearing individual interrupts).

Additionally, a 16 bit core control flag signal is provided (see \ttt{control\_flags} in Section \ref{sec:socinterface}). These flags may have a user-defined meaning and can also be read or written using the memory bus. Unlike most registers, these flags are \textit{not} affected by system resets, retaining their state.\\
For the bootloader code and UART programming utilities provided with EduSoC, only one of these flags (flag 0) has a meaning by default - it serves as an indicator that a program has been loaded into RAM. See Appendix \ref{sec:basic} for more information.

This peripheral has the following registers:\\
\begin{table}[H]
    \centering
    \begin{tabular}{|c|c|l|}\hline
        Address & Name & Reset Value \\\hline\hline
        \ttt{0x1B000000} & \ttt{SOCCON\_CONTROL} & \ttt{0x00000008}\footnotemark[1] \\
        \ttt{0x1B000010} & \ttt{SOCCON\_INT\_EN} & \ttt{0x00000000} \\
        \ttt{0x1B000020} & \ttt{SOCCON\_INT\_FLAGS} & \ttt{0x00000000} \\\hline
    \end{tabular}
    \caption{SoC Control Registers}
    \label{tab:soccon_regs}
\end{table}
\footnotetext[1]{Upper 16 bits are initially 0x0000, but retain their values during further resets.}

\subsubsection{\ttt{SOCCON\_CONTROL} Register}
\vspace{-3mm}
Address: \ttt{0x1B000000}\hfill
Reset value: \ttt{0x00000008}\footnotemark[1]\\[-1mm]
\begin{table}[H]
    \centering
    \renewcommand{\arraystretch}{1.1}
    \renewcommand{\extrarowheight}{-1.5mm}
    \begin{tabular}{|c||B|B|B|B|B|B|B|>{\arraybackslash}B|}\hline
        \rule{0pt}{12pt}Bit Range & \multicolumn{8}{c|}{Bits (highest to lowest)} \\\hline\hline
        \br{31:24} & \bd{R/W-0} & \bd{R/W-0} & \bd{R/W-0} & \bd{R/W-0} & \bd{R/W-0} & \bd{R/W-0} & \bd{R/W-0} & \bd{R/W-0} \\\cline{2-9}
        \rule{0pt}{12pt} & \multicolumn{8}{c|}{\ttt{control\_flags}[15:8]} \\\hline
        \br{23:16} & \bd{R/W-0} & \bd{R/W-0} & \bd{R/W-0} & \bd{R/W-0} & \bd{R/W-0} & \bd{R/W-0} & \bd{R/W-0} & \bd{R/W-0} \\\cline{2-9}
        \rule{0pt}{12pt} & \multicolumn{8}{c|}{\ttt{control\_flags}[7:0]} \\\hline
        \br{15:8} & \bd{U-0} & \bd{U-0} & \bd{U-0} & \bd{U-0} & \bd{U-0} & \bd{U-0} & \bd{U-0} & \bd{U-0} \\\cline{2-9}
        \rule{0pt}{12pt} & \ub & \ub & \ub & \ub & \ub & \ub & \ub & \ub \\\hline
        \br{7:0} & \bd{U-0} & \bd{U-0} & \bd{U-0} & \bd{U-0} & \bd{R/W-1} & \bd{R/W-0} & \bd{R/W-0} & \bd{R/W-0} \\\cline{2-9}
        \rule{0pt}{12pt} & \ub & \ub & \ub & \ub & \ttt{INTGEN} & \ttt{SOCRES} & \ttt{CORERES} & \ttt{COREHLT} \\\hline
        \multicolumn{9}{c}{} \\\hline
        \multicolumn{2}{|l}{\rule{0pt}{10pt}\footnotesize\textbf{Legend:}} & \multicolumn{2}{l}{\footnotesize R = Readable} & \multicolumn{2}{l}{\footnotesize W = Writable} & \multicolumn{3}{l|}{\footnotesize U = Unimplemented, read as 0} \\
        \multicolumn{2}{|l}{\rule{0pt}{10pt}\footnotesize -n = Initial Value} & \multicolumn{2}{l}{\footnotesize 1 = Set} & \multicolumn{2}{l}{\footnotesize 0 = Cleared} & \multicolumn{3}{l|}{\footnotesize x = Unknown} \\\hline
    \end{tabular}
    \renewcommand{\arraystretch}{1}
    \renewcommand{\extrarowheight}{0mm}
\end{table}
\vspace{-2mm}
\begin{itemize}[leftmargin=18mm,labelsep=3mm,parsep=1.5mm]
    \item[\footnotesize Bit 31-16] \ttt{control\_flags}: Core control flags, see description above. Reset retains values.
    \item[\footnotesize Bit 15-4] Unimplemented: Read as 0.
    \item[\footnotesize Bit 3] \ttt{INTGEN}: Interrupt global enable.\\{\footnotesize
    0 = No interrupts will be asserted to the core, irrespective of individual interrupt enable settings.\\
    1 = Individually enabled interrupts will be asserted to the core.}
    \item[\footnotesize Bit 2] \ttt{SOCRES}: SoC software reset.\\{\footnotesize
    0 = No effect.\\
    1 = SoC will be reset on the next clock cycle (including this register/bit).}
    \item[\footnotesize Bit 1] \ttt{CORERES}: CPU core reset.\\{\footnotesize
    0 = No effect.\\
    1 = CPU core reset line is asserted high (1), leading to a reset on the next CPU core clock cycle.}
    \item[\footnotesize Bit 0] \ttt{COREHLT}: CPU core halt.\\{\footnotesize
    0 = CPU core clock is running.\\
    1 = CPU core clock is halted. Note that this also prevents synchronous core resets from occurring.}
\end{itemize}

\subsubsection{\ttt{SOCCON\_INT\_EN} Register}
\vspace{-3mm}
Address: \ttt{0x1B000010}\hfill
Reset value: \ttt{0x00000000}\\[-1mm]
\begin{table}[H]
    \centering
    \renewcommand{\arraystretch}{1.1}
    \renewcommand{\extrarowheight}{-1.5mm}
    \begin{tabular}{|c||B|B|B|B|B|B|B|>{\arraybackslash}B|}\hline
        \rule{0pt}{12pt}Bit Range & \multicolumn{8}{c|}{Bits (highest to lowest)} \\\hline\hline
        \br{31:24} & \bd{R/W-0} & \bd{R/W-0} & \bd{R/W-0} & \bd{R/W-0} & \bd{R/W-0} & \bd{R/W-0} & \bd{R/W-0} & \bd{R/W-0} \\\cline{2-9}
        \rule{0pt}{12pt} & \multicolumn{8}{c|}{\ttt{INT\_ENABLE}[31:24]} \\\hline
        \br{23:16} & \bd{R/W-0} & \bd{R/W-0} & \bd{R/W-0} & \bd{R/W-0} & \bd{R/W-0} & \bd{R/W-0} & \bd{R/W-0} & \bd{R/W-0} \\\cline{2-9}
        \rule{0pt}{12pt} & \multicolumn{8}{c|}{\ttt{INT\_ENABLE}[23:16]} \\\hline
        \br{15:8} & \bd{R/W-0} & \bd{R/W-0} & \bd{R/W-0} & \bd{R/W-0} & \bd{R/W-0} & \bd{R/W-0} & \bd{R/W-0} & \bd{R/W-0} \\\cline{2-9}
        \rule{0pt}{12pt} & \multicolumn{8}{c|}{\ttt{INT\_ENABLE}[15:8]} \\\hline
        \br{7:0} & \bd{R/W-0} & \bd{R/W-0} & \bd{R/W-0} & \bd{R/W-0} & \bd{R/W-0} & \bd{R/W-0} & \bd{R/W-0} & \bd{R/W-0} \\\cline{2-9}
        \rule{0pt}{12pt} & \multicolumn{8}{c|}{\ttt{INT\_ENABLE}[7:0]} \\\hline
        \multicolumn{9}{c}{} \\\hline
        \multicolumn{2}{|l}{\rule{0pt}{10pt}\footnotesize\textbf{Legend:}} & \multicolumn{2}{l}{\footnotesize R = Readable} & \multicolumn{2}{l}{\footnotesize W = Writable} & \multicolumn{3}{l|}{\footnotesize U = Unimplemented, read as 0} \\
        \multicolumn{2}{|l}{\rule{0pt}{10pt}\footnotesize -n = Initial Value} & \multicolumn{2}{l}{\footnotesize 1 = Set} & \multicolumn{2}{l}{\footnotesize 0 = Cleared} & \multicolumn{3}{l|}{\footnotesize x = Unknown} \\\hline
    \end{tabular}
    \renewcommand{\arraystretch}{1}
    \renewcommand{\extrarowheight}{0mm}
\end{table}
\vspace{-2mm}
\begin{itemize}[leftmargin=18mm,labelsep=3mm,parsep=1.5mm]
    \item[\footnotesize Bit 31-0] \ttt{INT\_ENABLE}: Individual interrupt enable.\\{\footnotesize
    0 = Interrupt with the corresponding index will not be asserted to the core.\\
    1 = Interrupt with the corresponding index will be asserted to the core whenever it occurs.}
\end{itemize}
Each bit corresponds to one interrupt ID, e.\,g. bit 0 corresponds to interrupt ID 0.

When enabling an interrupt, if it has occurred in the past (i.\,e. its flag is 1, see \ttt{SOCCON\_INT\_FLAGS}), it will immediately be asserted to the core (assuming \ttt{INTGEN} = 1 in \ttt{SOCCON\_CONTROL}).

\newpage
\subsubsection{\ttt{SOCCON\_INT\_FLAGS} Register}
\vspace{-3mm}
Address: \ttt{0x1B000020}\hfill
Reset value: \ttt{0x00000000}\\[-1mm]
\begin{table}[H]
    \centering
    \renewcommand{\arraystretch}{1.1}
    \renewcommand{\extrarowheight}{-1.5mm}
    \begin{tabular}{|c||B|B|B|B|B|B|B|>{\arraybackslash}B|}\hline
        \rule{0pt}{12pt}Bit Range & \multicolumn{8}{c|}{Bits (highest to lowest)} \\\hline\hline
        \br{31:24} & \bd{R/C-0} & \bd{R/C-0} & \bd{R/C-0} & \bd{R/C-0} & \bd{R/C-0} & \bd{R/C-0} & \bd{R/C-0} & \bd{R/C-0} \\\cline{2-9}
        \rule{0pt}{12pt} & \multicolumn{8}{c|}{\ttt{INT\_FLAGS}[31:24]} \\\hline
        \br{23:16} & \bd{R/C-0} & \bd{R/C-0} & \bd{R/C-0} & \bd{R/C-0} & \bd{R/C-0} & \bd{R/C-0} & \bd{R/C-0} & \bd{R/C-0} \\\cline{2-9}
        \rule{0pt}{12pt} & \multicolumn{8}{c|}{\ttt{INT\_FLAGS}[23:16]} \\\hline
        \br{15:8} & \bd{R/C-0} & \bd{R/C-0} & \bd{R/C-0} & \bd{R/C-0} & \bd{R/C-0} & \bd{R/C-0} & \bd{R/C-0} & \bd{R/C-0} \\\cline{2-9}
        \rule{0pt}{12pt} & \multicolumn{8}{c|}{\ttt{INT\_FLAGS}[15:8]} \\\hline
        \br{7:0} & \bd{R/C-0} & \bd{R/C-0} & \bd{R/C-0} & \bd{R/C-0} & \bd{R/C-0} & \bd{R/C-0} & \bd{R/C-0} & \bd{R/C-0} \\\cline{2-9}
        \rule{0pt}{12pt} & \multicolumn{8}{c|}{\ttt{INT\_FLAGS}[7:0]} \\\hline
        \multicolumn{9}{c}{} \\\hline
        \multicolumn{2}{|l}{\rule{0pt}{10pt}\footnotesize\textbf{Legend:}} & \multicolumn{2}{l}{\footnotesize R = Readable} & \multicolumn{2}{l}{\footnotesize W = Writable} & \multicolumn{3}{l|}{\footnotesize U = Unimplemented, read as 0} \\
        \multicolumn{2}{|l}{\footnotesize C = Clearable} & \multicolumn{2}{l}{\rule{0pt}{10pt}\footnotesize -n = Initial Value} & \multicolumn{1}{l}{\footnotesize 1 = Set} & \multicolumn{2}{l}{\footnotesize 0 = Cleared} & \multicolumn{2}{l|}{\footnotesize x = Unknown} \\\hline
    \end{tabular}
    \renewcommand{\arraystretch}{1}
    \renewcommand{\extrarowheight}{0mm}
\end{table}
\vspace{-2mm}
\begin{itemize}[leftmargin=18mm,labelsep=3mm,parsep=1.5mm]
    \item[\footnotesize Bit 31-0] \ttt{INT\_FLAGS}: Interrupt occurred flags.\\{\footnotesize
    0 = Interrupt with the corresponding index has not occurred.\\
    1 = Interrupt with the corresponding index has occurred and will be asserted to the core if it is enabled.}
\end{itemize}
Each bit corresponds to one interrupt ID, e.\,g. bit 0 corresponds to interrupt ID 0.

Flags/bits in this register may be cleared (set to 0) using memory writes, which will clear the corresponding interrupts and stop them from being asserted until they reoccur. Attempting to set bits to 1 using memory writes will have no effect.